\documentclass{article}

\usepackage{amsthm}
\theoremstyle{definition}
\newtheorem{definition}{Definition}[section]

\begin{document}

\title{Discrete Notes}

\section*{2019-01-22}

\begin{definition}{Proposition}
  A true/false statement
\end{definition}

Examples:
\begin{itemize}
  \item "Today is Tuesday" is a bad proposition because it changes; currently true.
  \item $2+e=\pi$ is a proposition; false.
  \item $x+10=3x-5$ is not a proposition; varies with $x$.
  \item $x+10=3x-5$ when $x=10$ is a proposition (false)
  \item $x^2 + 3$ is positive is not a proposition
  \item $(x + y)^2 = x^2 + 2xy + y^2$ is not a proposition
\end{itemize}

\subsection*{Connectives}

\begin{definition}{Connective}
  Ways to build compound propositions from other propositions
\end{definition}

Note: Propositions are denoted p, q, r, \ldots

Types of Connectives:
\begin{itemize}
  \item Negation ($2+e=\pi$ becomes $2+e \neq \pi$); the negation of $p$ is $\neg p$
  \item Disjunction ($\lor$)
  \item Conjunction ($\land$)
  \item Exclusive Or ($\oplus$)
  \item Biconditional ($\leftrightarrow$)
  \item Conditional/implication ($\rightarrow$)
    \begin{itemize}
      \item ``If $p$ (hypothesis) then $q$ (conclusion)''
      \item ``$p$ implies $q$''
    \end{itemize}
\end{itemize}

\subsection*{Truth Tables}
\begin{tabular}{c | c | c | c | c | c | c}
  $p$ & $q$ & $p \lor q$ & $p \land q$ & $p \oplus q$ & $p \leftrightarrow q$ & $p \rightarrow q$ \\
  \hline
  T & T & T & T & F & T & T \\
  \hline
  T & F & T & F & T & F & F \\
  \hline
  F & T & T & F & T & F & T \\
  \hline
  F & F & F & F & F & T & T
\end{tabular}

\section*{2019-01-23}

``Anyone getting at least 90\% in this class gets an A.'' is an implication. Therefore there must be a hypothesis and a conclusion. ``\textbf{If} someone gets at least 90\%, \textbf{then} they get an A.'' 

\begin{tabular}{c | c | c | l }
  Percentage & Grade & Truth Value & Notes \\
  \hline
  93 & A & T & This is true. \\
  88 & A & T & The implication doesn't dictate grade here. \\
  70 & C & T \\
  95 & B & F & The conclusion wasn't met but the hypothesis was. \\

\end{tabular}

\begin{quote}
``You can't win if you don't play.''
\end{quote}

If $p \rightarrow q$, what is $p$ and what is $q$?

$p$: If I don't play,

$q$: then I won't win.

\begin{quote}
You have to pay to play.
\end{quote}

This is false when playing happened without paying. 

\begin{quote}
  If $x=2$ then $x^2=4$. (true) \newline
  If $x^2=4$ then $x=2$. (false)
\end{quote}

\begin{tabular}{c|c|c|c|c}
  $x$ & $x=2$ & $x^2=4$ & $x=2 \rightarrow x^2=4$ & $x^2=4 \rightarrow x=2$ \\
  \hline
  2 & T & T & T & T \\
  5 & F & F & T & T \\
  -2 & F & T & T & F
\end{tabular}

Note in this case there is no fourth row, because there is no x value to make that statement true.

Expanding on previous:

\begin{tabular}{c | c | c | c | c | c | c | c | c | c | c | c}
  $p$ & $q$ & $p \lor q$ & $p \land q$ & $p \oplus q$ & $p \leftrightarrow q$ & $p \rightarrow q$ & $\neg p$ & $\neg q$ & $q \rightarrow p$ & $\neg p \rightarrow \neg q$ & $\neg q \rightarrow \neg p$ \\
  \hline
  T & T & T & T & F & T & T & F & F & T & T & T \\
  \hline
  T & F & T & F & T & F & F & F & T & T & T & F\\
  \hline
  F & T & T & F & T & F & T & T & F & F & F & T\\
  \hline
  F & F & F & F & F & T & T & T & T & T & T & T
\end{tabular}

$q \rightarrow p$ is the converse; $\neg p \rightarrow \neg q$ is the inverse; $\neg q \rightarrow \neg p$ is the contrapositive. 

\subsubsection*{Example: $(p \land \neg q) \rightarrow (q \lor r)$}
\begin{tabular}{c|c|c|c|c|c|c}
  $p$ & $q$ & $r$ & $\neg q$ & $p \land \neg q$ & $q \lor r$ & $(p \land \neg q) \rightarrow (q \lor r)$ \\
  \hline
  T & T & T & F & F & T & T \\
  T & T & F & F & F & T & T \\
  T & F & T & T & T & T & T \\
  T & F & F & T & T & F & F \\
  \hline
  F & T & T & F & F & T & T \\
  F & T & F & F & F & T & T \\
  F & F & T & T & F & T & T \\
  F & F & F & T & F & F & T
\end{tabular}

\section*{Bit Strings}
Ex: 001101001 is a bitstring of length 9.

$001101001 \lor 101010101 = 101111101$

Note: $p \leftrightarrow q$ is not the same as $\neg (p \oplus q)$ even though they are equivalent. They are different propositions, though their truth values are always the same. They are not equal, but are logically equivalent: $p \rightarrow q \equiv \neg q \rightarrow \neg p$.

\begin{definition}{Tautology}
  A compound proposition that always evaluates to true; for example, $p \lor \neg p$. (Opposite of a contradiction)
\end{definition}

\begin{definition}{Contradiction}
  A compound proposition that always evaluates to false; for example, $p \land \neg p$. (Opposite of a tautology)
\end{definition}

\end{document}
