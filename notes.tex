
\documentclass{article}

\usepackage{amsthm}
\usepackage{amsmath}
\usepackage{amsfonts}
\theoremstyle{definition}
\newtheorem{definition}{Definition}[section]

\begin{document}

\title{Discrete Notes}

\section*{2019-01-22}

\begin{definition}{Proposition}
  A true/false statement
\end{definition}

Examples:
\begin{itemize}
  \item "Today is Tuesday" is a bad proposition because it changes; currently true.
  \item $2+e=\pi$ is a proposition; false.
  \item $x+10=3x-5$ is not a proposition; varies with $x$.
  \item $x+10=3x-5$ when $x=10$ is a proposition (false)
  \item $x^2 + 3$ is positive is not a proposition
  \item $(x + y)^2 = x^2 + 2xy + y^2$ is not a proposition
\end{itemize}

\subsection*{Connectives}

\begin{definition}{Connective}
  Ways to build compound propositions from other propositions
\end{definition}

Note: Propositions are denoted p, q, r, \ldots

Types of Connectives:
\begin{itemize}
  \item Negation ($2+e=\pi$ becomes $2+e \neq \pi$); the negation of $p$ is $\neg p$
  \item Disjunction ($\lor$)
  \item Conjunction ($\land$)
  \item Exclusive Or ($\oplus$)
  \item Biconditional ($\leftrightarrow$)
  \item Conditional/implication ($\rightarrow$)
    \begin{itemize}
      \item ``If $p$ (hypothesis) then $q$ (conclusion)''
      \item ``$p$ implies $q$''
    \end{itemize}
\end{itemize}

\subsection*{Truth Tables}
\begin{tabular}{c | c | c | c | c | c | c}
  $p$ & $q$ & $p \lor q$ & $p \land q$ & $p \oplus q$ & $p \leftrightarrow q$ & $p \rightarrow q$ \\
  \hline
  T & T & T & T & F & T & T \\
  \hline
  T & F & T & F & T & F & F \\
  \hline
  F & T & T & F & T & F & T \\
  \hline
  F & F & F & F & F & T & T
\end{tabular}

\section*{2019-01-23}

``Anyone getting at least 90\% in this class gets an A.'' is an implication. Therefore there must be a hypothesis and a conclusion. ``\textbf{If} someone gets at least 90\%, \textbf{then} they get an A.'' 

\begin{tabular}{c | c | c | l }
  Percentage & Grade & Truth Value & Notes \\
  \hline
  93 & A & T & This is true. \\
  88 & A & T & The implication doesn't dictate grade here. \\
  70 & C & T \\
  95 & B & F & The conclusion wasn't met but the hypothesis was. \\

\end{tabular}

\begin{quote}
``You can't win if you don't play.''
\end{quote}

If $p \rightarrow q$, what is $p$ and what is $q$?

$p$: If I don't play,

$q$: then I won't win.

\begin{quote}
You have to pay to play.
\end{quote}

This is false when playing happened without paying. 

\begin{quote}
  If $x=2$ then $x^2=4$. (true) \newline
  If $x^2=4$ then $x=2$. (false)
\end{quote}

\begin{tabular}{c|c|c|c|c}
  $x$ & $x=2$ & $x^2=4$ & $x=2 \rightarrow x^2=4$ & $x^2=4 \rightarrow x=2$ \\
  \hline
  2 & T & T & T & T \\
  5 & F & F & T & T \\
  -2 & F & T & T & F
\end{tabular}

Note in this case there is no fourth row, because there is no x value to make that statement true.

Expanding on previous:

\begin{tabular}{c | c | c | c | c | c | c | c | c | c | c | c}
  $p$ & $q$ & $p \lor q$ & $p \land q$ & $p \oplus q$ & $p \leftrightarrow q$ & $p \rightarrow q$ & $\neg p$ & $\neg q$ & $q \rightarrow p$ & $\neg p \rightarrow \neg q$ & $\neg q \rightarrow \neg p$ \\
  \hline
  T & T & T & T & F & T & T & F & F & T & T & T \\
  \hline
  T & F & T & F & T & F & F & F & T & T & T & F\\
  \hline
  F & T & T & F & T & F & T & T & F & F & F & T\\
  \hline
  F & F & F & F & F & T & T & T & T & T & T & T
\end{tabular}

$q \rightarrow p$ is the converse; $\neg p \rightarrow \neg q$ is the inverse; $\neg q \rightarrow \neg p$ is the contrapositive. 

\subsubsection*{Example: $(p \land \neg q) \rightarrow (q \lor r)$}
\begin{tabular}{c|c|c|c|c|c|c}
  $p$ & $q$ & $r$ & $\neg q$ & $p \land \neg q$ & $q \lor r$ & $(p \land \neg q) \rightarrow (q \lor r)$ \\
  \hline
  T & T & T & F & F & T & T \\
  T & T & F & F & F & T & T \\
  T & F & T & T & T & T & T \\
  T & F & F & T & T & F & F \\
  \hline
  F & T & T & F & F & T & T \\
  F & T & F & F & F & T & T \\
  F & F & T & T & F & T & T \\
  F & F & F & T & F & F & T
\end{tabular}

\subsection*{Bit Strings}
Ex: 001101001 is a bitstring of length 9.

$001101001 \lor 101010101 = 101111101$

Note: $p \leftrightarrow q$ is not the same as $\neg (p \oplus q)$ even though they are equivalent. They are different propositions, though their truth values are always the same. They are not equal, but are logically equivalent: $p \rightarrow q \equiv \neg q \rightarrow \neg p$.

\begin{definition}{Tautology}
  A compound proposition that always evaluates to true; for example, $p \lor \neg p$. (Opposite of a contradiction)
\end{definition}

\begin{definition}{Contradiction}
  A compound proposition that always evaluates to false; for example, $p \land \neg p$. (Comparable to a tautology) If a proposition is not a contradiction, it is then \textit{satisfyable}.
\end{definition}

\section*{2019-01-24}

\subsection*{Tautologies}

Which of these is a tautology? (three are: All but the last)

\begin{itemize}
  \item $[(p \rightarrow q) \land (r \rightarrow q)] \rightarrow [(p \land r) \rightarrow q]$
  \item $[(p \rightarrow q) \lor  (r \rightarrow q)] \rightarrow [(p \land r) \rightarrow q]$
  \item $[(p \rightarrow q) \land  (r \rightarrow q)] \rightarrow [(p \lor r) \rightarrow q]$
  \item $[(p \rightarrow q) \lor  (r \rightarrow q)] \rightarrow [(p \lor r) \rightarrow q]$ (Can be disproven with $p = \text{True}, q = \text{False}, r = \text{False}$)
\end{itemize}

Connectives are binary operators. This means that $a \land b \land c$ can be
read as either $(a \land b) \land c$ or $a \land (b \land c)$. It doesn't
matter as long as they're all the same operation. 

However, if they are \textit{not} all the same operation, this doesn't work: $a \land b \lor c$ has two non equivalent possibilities, $(a \land b) \lor c$ and $a \land (b \lor c)$.

\subsection*{Logical Equivalences}

\begin{definition}{Logically Equivalent}
  $S(p, q, r, \ldots)$ is said to be logically equivalent to $T(p, q, r, \ldots)$ whenever $S \leftrightarrow T$ is a tautology. This is written $S \equiv T$.
\end{definition}

Note: In all of the following cases, \textit{not} ($\neg$ is the first operation to be applied.

\begin{itemize}
  \item Associative Laws:
    $(a \land b) \land c \equiv a \land (b \land c)$ and
    $(a \lor b) \lor c \equiv a \lor (b \lor c)$, which means that parentheses
    can be dropped when order doesn't matter.
  \item Commutative Laws:
    $a \lor b \equiv b \lor a$ and $a \land b \equiv b \land a$
  \item Double Negation Law:
    $\neg(\neg a) \equiv a$
  \item De Morgan's Laws:
    $\neg(a \lor b) \equiv \neg a \land \neg b$ and
    $\neg(a \land b) \equiv \neg a \lor \neg b$
  \item Identity Laws:
    $a \land T \equiv a$ and
    $a \lor F \equiv a$
  \item Domination Laws:
    $a \land F \equiv F$ and
    $a \lor T \equiv T$
  \item Idempotent Laws
    $a \lor a \equiv a$ and
    $a \land a \equiv a$
  \item Distributive Laws:
    $a \land (b \lor c) \equiv (a \lor b) \land (a \lor c)$ and
    $a \lor (b \land c) \equiv (a \land b) \lor (a \land c)$
  \item Apparently-no-name Laws:
    $p \leftrightarrow q \equiv \neg (p \oplus q)$ and
    $p \rightarrow q \equiv \neg p \lor q$
\end{itemize}

Aside: $(p \land q) \lor (p \rightarrow q) \equiv p \rightarrow q$

$$(p \land q) \lor (p \rightarrow q)$$
$$\equiv (p \land q) \lor \neg p \lor q$$
$$\equiv \neg p \lor (p \land q) \lor q$$
$$\equiv [(\neg p \lor p) \land (\neg p \lor q)] \lor q$$
$$\equiv (\neg p \lor q) \lor q)$$
$$\equiv \neg p \lor q$$
$$\equiv p \rightarrow q$$
 
\section*{2019-01-28}

(Proved logical equivalences)

\section*{2019-01-31}

\subsection*{Chapter 2: Predicates and Quantifiers}

\begin{definition}{Propositional Function}
  A function that gives a proposition for every $x$ value
\end{definition}

\begin{definition}{Universe of Discourse}
  All allowable values for $x$.
\end{definition}

\begin{definition}{Quantifier}
  A way of making a proposition out of a collection of propositional functions.
\end{definition}

\begin{definition}{Universal Quantification}
  The statement that $p(x)$ is true for all $x$ in the universe of discourse,
  noted with $\forall$.
\end{definition}

For example, the statement $x=2 \rightarrow x=4$ is true for all x, and as such
is a universally qualified statement, notated as
$$\forall x (x=2 \rightarrow x=4)$$

At the other extreme from universal quantification is existential
quantification:

\begin{definition}{Existential Quantification}
  $p(x)$ is true for at least one value of $x$, noted with $\exists$.
\end{definition}

$\exists x (x^2=4 \rightarrow x=2)$ can then be read as ``There exists an x
such that $x^2=4$ implies $x=2$'', or as ``There is an $x$ for which $x^2=4$
implies $x=2$.

For example: $\exists x (x^2-x \geq 0$ is false if the universe consists of all reals, true if the universe exists of integers. 

For example, $U={1,3,5,7}$: $\forall x (x^2 < 10x)$ is true (and can be written
as $1^2 < 10 \cdot 1 \land 3^2 < 10 \cdot 3 \land \ldots)$, as is
$\exists x (x^2<4x)$, which can be written as $1^2 < 4 \cdot 1 \lor 3^2 < 4 \cdot
3 \lor \ldots$. 
 
\section*{2019-02-04}
\subsection*{Review:}
$x > 0$ is a not a proposition; instead, it is a propositional function. This
means that for any x value we get an actual proposition; e.g. $P(5)$ is true,
and $P(8)$ is false.

Quantification: many types; we covered two (universal ($\forall$) and
existential ($\exists$)). As an example, $\forall x \forall y ((x+y)^2 = x^2 + 2xy + y^2)$.

\subsection*{New stuff:}

$\forall x \exists y (x+y > 0)$: ``For every $x$ there is a $y$ for which $x+y>0$'', which is true. However, $\exists y \forall x (x+y>0)$: ``There is a $y$ so that for every $x$, $x+y > 0$'', which is false.

Another quantification is the existence of a unique value, $\exists!$. For example, $\exists! x (x^3=8)$; ``There is a unique $x$ for which $x^3=8$'', which is true. However, $\exists! x ( x^2 = 4 )$ is false, as is $\exists! x ( x^2 = -4 )$.

\begin{definition}{Increasing Function}
  $\forall x \forall y ( y > x \rightarrow f(y) > f(x))$
\end{definition}

\begin{definition}{Bertram's Postulate}
  (not actually a postulate?) -- There is always a prime between a number 
  (positive integer greater than or equal to two)  and its double; that is,
  $\forall n (n>2 \rightarrow \exists p (\text{p is prime} \land n<p<2n))$.
\end{definition}

``Ex: If an increasing continuous function starts negative and ends positive,
then it must be zero somewhere in between.'' This can be written as
$\forall f ([ f \text{is increasing} \land f \text{is continuous} \land
\exists x_1 (f(x_1) < 0) \land \exists x_2 (f(x_2) > 0)] \rightarrow
\exists c (x_1 < c < x_2 \land f(c) = 0))$

\begin{definition}{Bound}
  A variable that has been quantified is called bound.
\end{definition}

For example, in this propositional function of $x$: $\exists y (x + y > 0)$,
$y$ is bound but not $x$. As such, values are added for $y$ such as
$P(10): \exists y (10+y > 0)$, or $P(-100): \exists y (-100 + y > 0)$. 

\section*{2019-02-05}

\subsection*{Goldbach's Conjecture}

\begin{definition}{Goldbach's Conjecture}
  Every even number 4 or larger is the sum of two primes; that is,
  $\forall n ( ( n \text{ is even } \land n \geq 4 ) \rightarrow 
  \exists p \exists q ( p \text{ is prime } \land q \text{ is prime }
  \land n = p + q ) )$
\end{definition}

\begin{definition}{The Mean Value Theorem}
  If $f(x)$ is continous on $[a,b]$ and differentiable on $(a,b)$ then for
  some $c$ with $a<c<b$, $$f'(c) = \frac{f(b)-f(a)}{b-a}$$
  That is, $\forall f \forall a \forall b ([a < b \land f \text{ is
  continuous on } [a,b] \land f \text{ is differentiable on } (a,b)] \to
  \exists c (a < c < b \land f'(c) = \frac{f(b)-f(a)}{b-a}) )$.
\end{definition}

\section*{2019-02-07}
\subsection*{Terminology}
\begin{definition}{Theorem}
  Something that has a proof. Or, an \textit{important} theorem.
\end{definition}
\begin{definition}{Lemma}
  A type of theorem
\end{definition}
\begin{definition}{Corrolary}
  A type of theorem that is a direct application of another theorem.
\end{definition}

\section*{2019-02-11}
\subsection*{Proof Techniques}
\begin{enumerate}
  \item Direct Proof: Assume $P(x)$ is true; show $Q(x)$ must be true.
  \item Indirect Proof: Anything Else
    \begin{enumerate}
      \item Contraposition: A Direct proof that $\neg Q(x) \to \neg P(x)$
      \item
    \end{enumerate}
\end{enumerate}

For statement $P \to Q$:
\begin{itemize}
  \item The converse is $Q \to P$
  \item The contrapositive is $\neg Q \to \neg P$
  \item The inverse is $\neg P \to \neg Q$, which is equivalent to the
    converse (It is the contrapositive of the converse)
\end{itemize}

\subsection*{Examples}

If $n$ is even, then $9n-5$ is odd:
\begin{proof}
  Suppose $n$ is even, say $n-2k$ for some integer $k$. Then
  \begin{align*}
    9n-5 &= 9\times2k -5\\
    &= 18k-5\\
    &= 2(9k-3) + 1
  \end{align*}
  and so $9n-5$ is odd.
\end{proof}

If $9n-5$ is odd, then $n$ is even. (This fails to work directly)
\begin{proof}
  If $n$ is odd, then $n=2m+1$ for some integer $m$. Now $9n-5=9(2m-1)-5=18m+4=2(9m+2)$, which is even.
  So $9n-5$ is even.
\end{proof}

If $x \cdot y > 100$ then $x > 10$ or $y > 10$:
Can't be proven---this isn't true!

Fixed version: If $x, y$ are positive and $x \cdot y > 100$, then $x > 10$ or
$y > 10$.

Quantified: $\forall x \forall y ( ( x > 0 \land y > 0 \land x \cdot y > 100 )
\to ( x > 10 \lor y > 10)$

Contrapositive: $\forall x \forall y ( ( x \leq 10 \land y \leq 10 ) \to
( x \leq 0 \lor y \leq 0 \lor x \cdot y \leq 100) ) $

Aside: $p \to q \equiv (p \land \neg q) \to F$.

\begin{proof}
\end{proof}

If $x+y > 100$ then $x > 50$ or $y > 50$:
\begin{proof}
  By way of contraposition: Suppose $x \leq 50$ and $y \leq 50$. Then
  $x + y \leq 50 + 50 = 100$, the negation of $x+y > 100$.
\end{proof}

If the square of an integer is even, then the integer is even.
\begin{proof}
  By way of contraposition: 
  Suppose
\end{proof}

Anything you can prove directly or by contraposition, you can prove by
contradiction.

\section*{2019-02-18}
\subsection*{Mistakes to avoid in proofs}
\begin{itemize}
  \item Arguing by example: ``The sum of two numbers is even, because eight is even, and 20 is even, and 20+8 is even.''
  \item Jumping to conclusions: Skip a necessary step of the proof, even if it seems ``obvious''
  \item Overworking notation: Don't let $x$ represent two different things.
  \item Begging the question: Assuming something that wasn't in the question.
    ``If $n^2$ is even then $n$ is even: ``Suppose $n^2$ is even. Then $n^2=4k^2$\dots''
     no it's not!''
\end{itemize}

\subsection*{Sets}
\begin{definition}{Set}
  A set is an unordered collection of things.
\end{definition}

Notation: $\{1,2,5,11,13\}$ is the set containing 1, 2, 5, 11, and 13. Sets are
unordered, so $\{1,2,5,11,13\}=\{5,1,13,2,11\}$. Two sets are equal when they
contain exactly the same things. Repetition is ignored/does not matter;
$\{1, 2, 5, 11, 13\}=\{1,2,2,5,5,5,11,11,13\}$. Sets are very general:
$\{1,8,B,q,\{1,B,C\}\}$.

We say 8 is an \textbf{element} or a \textbf{member} of the set.

Commonly , $S,T,A,B,C$ are sets; $8 \in S$, $8 \notin T$.

\section*{2019-02-19}
A \textbf{set} is an unordered collection of ``things'', the elements of the set.

$\{1,2,3\}=\{3,1,2\}=\{1,1,1,2,2,3\}$ -- Order doesn't matter; repetition is
ignored.

$\{1,2,3,4,5,6,7,8,9,10\} = \{x | x \text{ is an integer } \land x \geq 1 \land
x \leq 10 \}$ (``Set constructor notation'')

$S={x|P(x)}$ means $S$ is the set of all x for which the propositional function
$P(x)$ is true.

$S=T$ means that $\forall x (x \in S \leftrightarrow x \in T)$ is true.

\subsection*{Special Sets}

$\mathbb{Z}$: The set of all integers

$\mathbb{Z}=\{\dots,-2,-1,0,1,2,\dots\}$

$\mathbb{Q}$: The set of rational numbers

$\mathbb{Q} = \{ \frac{m}{n} | m,n \in \mathbb{Z} \land n \neq 0 \}$

$\mathbb{R}$: The Reals

$\mathbb{C}$: The Complex

$\mathbb{C}= \{ x + iy | x,y \in \mathbb{R} \}$

$\mathbb{N}$: Natural Numbers

$\mathbb{N}=\{0,1,2,3,\dots\}$

$\mathbb{Z}^+ = \{1,2,3,\dots\}$

\subsection*{Subsets and other sets}

$S$ is a subset of $T$, denoted $S \subset T$, means that 
$\forall x (x \in S \to x \in T)$ is true. 

$\mathbb{Z}^+ \subset \mathbb{N} \subset \mathbb{Z} \subset
\mathbb{Q} \subset \mathbb{R} \subset \mathbb{C}$

The empty set, $\emptyset = \{\}$ has no elements, and is a subset of every set.

Every set is also a subset of itself. 

NOTE: $\{0\} \neq \emptyset = \{\}$, and $\{\emptyset\} \neq \emptyset$.

$|\{0, \{\emptyset\}, \emptyset \}| = 3$ (``The cardinality is three.'')

$P(S)$ is the \textbf{Power Set} of $S$, the set of all subsets of $S$.

For example, if $S=\{1,2\}$, then $P(S)=\{\emptyset, \{1\}, \{2\}, \{1,2\}\}$.

The magnitude of a power set $|P(S)| = 2^{|S|}$.

What can be in a set? Is ``The set of all sets'' a valid set? It contains
itself, by definition\dots.



\section*{2019-02-20}

\subsection*{Tuples}

An ordered $n$-tuple is an expression like $(a_1, a_2, \dots, a_n)$ where both
order and repetition matter.

An example would be an ordered pair: $(5,8)$ or $(3,3)$. 

$A \times B$ (``$A$ cross $B$''), the cartesian product of $A$ and $B$ is
$\{ (x,y) | x \in A \land y \in B\}$.

Ex: $\{1,2\}\times\{3,4\}=\{(1,2),(1,3),(1,4),(2,2),(2,3),(2,4)\}$

The $xy$-plane is $\mathbb{R}\times\mathbb{R}$.

$|A \times B| = |A| \cdot |B|$.

The \textbf{union} of two sets
$A \cup B = \{ x | x \in A \lor x \in B \}$.

The \textbf{intersection} of two sets
$A \cap B = \{ x | x \in A \land x \in B \}$.

The \textbf{symmetric difference} of two sets
$A \oplus B = \{ x | x \in A \oplus x \in B \}$.

The \textbf{set difference} of two sets
$A - B = \{x | x \in A \land x \notin B \}$.

$A \oplus B = (A-B)\cup(B-A)$

The \textbf{complement} of a set
$\bar{A}=\{x|x \notin A\}$.

\subsection*{Set Identities}

Commutative Laws:\\
$A \cup B = B \cup A$\\
$A \cap B = B \cap A$\\

Identity Laws:\\
$A \cup \emptyset = A$\\
$A \cap U = A$\\

Domination Laws:\\
$A \cap \emptyset = \emptyset$ \\
$A \cup U = U$\\

Idempotent Laws:\\
$A \cup A = A$\\
$A \cap A = A$\\

Associative Laws:\\
$A \cup (B \cup C) = (A \cup B) \cup C$\\
$A \cap (B \cap C) = (A \cap B) \cap C$\\

Distributive Laws:\\
$A \cap (B \cup C) = (A \cap B) \cup (A \cap C)$\\
$A \cup (B \cap C) = (A \cup B) \cap (A \cup C)$\\

Double Complement Law:\\
$\bar{(\bar{A})}=A$\\

De Morgan's Laws:\\
$\bar{A \cup B} = \bar{A} \cap \bar{B}$\\
$\bar{A \cap B} = \bar{A} \cup \bar{B}$\\

\section*{2019-02-26}

If $A \oplus B = A \oplus C$, does B have to equal C (symmetric difference)? (yes)
If $A \times B = A \times C$, does B have to equal C? (no -- A could be the 
empty set, for which $A \times B = A \times C = \emptyset$ for any sets $B,C$.)\\

Proof by truth table: For all rows where $A \oplus B = A \oplus C$, $B=C$.

\begin{tabular}{ c|c|c|c|c }
  $A$ & $B$ & $C$ & $A \oplus B$ & $A \oplus C$\\
  \hline
  1 & 1 & 1 & 0 & 0\\
  1 & 1 & 0 & 0 & 1\\
  1 & 0 & 1 & 1 & 0\\
  1 & 0 & 0 & 1 & 1\\
  \hline
  0 & 1 & 1 & 1 & 1\\
  0 & 1 & 0 & 1 & 0\\
  0 & 0 & 1 & 0 & 1\\
  0 & 0 & 0 & 0 & 0
\end{tabular}\\

\begin{proof}
  Let $x \in B$. If $x \in A \oplus B$ then $x \notin A$. Since $A \oplus B =
  A \oplus C$, $x \in A \oplus C$. Since $x \notin A$, $x \in C$. If $x \notin
  A \oplus B$ but $x \in B$ then $x \in A$. Again, $x \notin A \oplus B =\
  A \oplus C$, so $x \in A$, $x \notin A \oplus C$. Hence, again, $x \in C$.
  Consequently, $B \subset C$. By symmetry, $C \subset B$, so $B=C$.
\end{proof}

If $A \neq \emptyset$ and $A \times B = A \times C$, then $B=C$.
\begin{proof}
  Let $x \in B$. Since $A \neq \emptyset$, there is some element $a \in A$.
  Now $(a,x) \in A \times B$, but $A \times B = A \times C$ so
  $(a,x) \in A \times C$ as well. Hence $x \in C$. So $B \subset C$.
  By symmetry, $C \subset B$, so $B = C$.
\end{proof}

\subsection*{2.3. Functions}

Given two sets $A$ and $B$, a \textbf{function} $f: A \to B$ is an assignment
of a unique element of $B$ to each element of $A$.

$A$ is called the \textbf{domain} of $f$.

$B$ is the \textbf{codomain} (\textit{not} the range!).

The \textbf{range} $\{f(x) | x \in A\}$. For example, the $\sin$ function
has a domain of the reals, the codomain of the reals, and the range $[0,1]$.

Unlike algebra, $f(x)=\sqrt{x}$ is not a function from $\mathbb{R}$ to
$\mathbb{R}$, since not every value in the domain has a corresponding value
in the codomain, and values that \textit{are} defined in the codomain have
two values. Similarly, $g(x) = \frac{x}{x+1}$ is not a function from
$\mathbb{R}$ to $\mathbb{R}$, since we can't divide by zero when $x=-1$.


\section{2019-02-27}

A function $f: S \to T$ is an assignment of a unique element of $T$ to each
element of $S$. We wrute this $f(x)=y$, where $x$ is the ``argument'' of $f$
and $y$ is the ``image of $x$'', and $x$ is the ``preimage'' of $y$.

$f$ is called \textbf{injective} (``one-to-one'') if $f(x) \neq f(y)$ when
$x \neq y$, for $x,y \in S$. 

$f$ is called \textbf{surjective} (``onto'') if the range of $f: S \to T$ is
all of $T$. 

Example: $f(x)=x^2$ is not one-to-one or onto in $\mathbb{R} \to \mathbb{R}$,
because $f(x)=f(-2)$, and $f(x)=-2$ has no solutions. However, in the domain
and codomain of the positive reals, both are true. If both the domain and 
codomain are natural numbers, then the function is one-to-one but not onto.

A function that is both injective and surjective is called \textbf{bijective}.



\section*{2019-03-06}

\subsection*{Sequences and Summations (Series)}

A sequence is a function from the natural numbers to some set $A$: $f: \mathbb{N} \to A$.


\end{document}
