\documentclass{article}

\usepackage{amsthm}
\theoremstyle{definition}
\newtheorem{definition}{Definition}[section]

\begin{document}

\title{Discrete Notes}

\section*{2019-01-22}

\begin{definition}{Proposition}
  A true/false statement
\end{definition}

Examples:
\begin{itemize}
  \item "Today is Tuesday" is a bad proposition because it changes; currently true.
  \item $2+e=\pi$ is a proposition; false.
  \item $x+10=3x-5$ is not a proposition; varies with $x$.
  \item $x+10=3x-5$ when $x=10$ is a proposition (false)
  \item $x^2 + 3$ is positive is not a proposition
  \item $(x + y)^2 = x^2 + 2xy + y^2$ is not a proposition
\end{itemize}

\subsection*{Connectives}

\begin{definition}{Connective}
  Ways to build compound propositions from other propositions
\end{definition}

Note: Propositions are denoted p, q, r, \ldots

Types of Connectives:
\begin{itemize}
  \item Negation ($2+e=\pi$ becomes $2+e \neq \pi$); the negation of $p$ is $\neg p$
  \item Disjunction ($\lor$)
  \item Conjunction ($\land$)
  \item Exclusive Or ($\oplus$)
  \item Biconditional ($\leftrightarrow$)
  \item Conditional/implication ($\rightarrow$)
    \begin{itemize}
      \item ``If $p$ (hypothesis) then $q$ (conclusion)''
      \item ``$p$ implies $q$''
    \end{itemize}
\end{itemize}

\subsection*{Truth Tables}
\begin{tabular}{c | c | c | c | c | c | c}
  $p$ & $q$ & $p \lor q$ & $p \land q$ & $p \oplus q$ & $p \leftrightarrow q$ & $p \rightarrow q$ \\
  \hline
  T & T & T & T & F & T & T \\
  \hline
  T & F & T & F & T & F & F \\
  \hline
  F & T & T & F & T & F & T \\
  \hline
  F & F & F & F & F & T & T
\end{tabular}

\section*{2019-01-23}

``Anyone getting at least 90\% in this class gets an A.'' is an implication. Therefore there must be a hypothesis and a conclusion. ``\textbf{If} someone gets at least 90\%, \textbf{then} they get an A.'' 

\begin{tabular}{c | c | c | l }
  Percentage & Grade & Truth Value & Notes \\
  \hline
  93 & A & T & This is true. \\
  88 & A & T & The implication doesn't dictate grade here. \\
  70 & C & T \\
  95 & B & F & The conclusion wasn't met but the hypothesis was. \\

\end{tabular}

\begin{quote}
``You can't win if you don't play.''
\end{quote}

If $p \rightarrow q$, what is $p$ and what is $q$?

$p$: If I don't play,

$q$: then I won't win.

\begin{quote}
You have to pay to play.
\end{quote}

This is false when playing happened without paying. 

\begin{quote}
  If $x=2$ then $x^2=4$. (true) \newline
  If $x^2=4$ then $x=2$. (false)
\end{quote}

\begin{tabular}{c|c|c|c|c}
  $x$ & $x=2$ & $x^2=4$ & $x=2 \rightarrow x^2=4$ & $x^2=4 \rightarrow x=2$ \\
  \hline
  2 & T & T & T & T \\
  5 & F & F & T & T \\
  -2 & F & T & T & F
\end{tabular}

Note in this case there is no fourth row, because there is no x value to make that statement true.

Expanding on previous:

\begin{tabular}{c | c | c | c | c | c | c | c | c | c | c | c}
  $p$ & $q$ & $p \lor q$ & $p \land q$ & $p \oplus q$ & $p \leftrightarrow q$ & $p \rightarrow q$ & $\neg p$ & $\neg q$ & $q \rightarrow p$ & $\neg p \rightarrow \neg q$ & $\neg q \rightarrow \neg p$ \\
  \hline
  T & T & T & T & F & T & T & F & F & T & T & T \\
  \hline
  T & F & T & F & T & F & F & F & T & T & T & F\\
  \hline
  F & T & T & F & T & F & T & T & F & F & F & T\\
  \hline
  F & F & F & F & F & T & T & T & T & T & T & T
\end{tabular}

$q \rightarrow p$ is the converse; $\neg p \rightarrow \neg q$ is the inverse; $\neg q \rightarrow \neg p$ is the contrapositive. 

\subsubsection*{Example: $(p \land \neg q) \rightarrow (q \lor r)$}
\begin{tabular}{c|c|c|c|c|c|c}
  $p$ & $q$ & $r$ & $\neg q$ & $p \land \neg q$ & $q \lor r$ & $(p \land \neg q) \rightarrow (q \lor r)$ \\
  \hline
  T & T & T & F & F & T & T \\
  T & T & F & F & F & T & T \\
  T & F & T & T & T & T & T \\
  T & F & F & T & T & F & F \\
  \hline
  F & T & T & F & F & T & T \\
  F & T & F & F & F & T & T \\
  F & F & T & T & F & T & T \\
  F & F & F & T & F & F & T
\end{tabular}

\subsection*{Bit Strings}
Ex: 001101001 is a bitstring of length 9.

$001101001 \lor 101010101 = 101111101$

Note: $p \leftrightarrow q$ is not the same as $\neg (p \oplus q)$ even though they are equivalent. They are different propositions, though their truth values are always the same. They are not equal, but are logically equivalent: $p \rightarrow q \equiv \neg q \rightarrow \neg p$.

\begin{definition}{Tautology}
  A compound proposition that always evaluates to true; for example, $p \lor \neg p$. (Opposite of a contradiction)
\end{definition}

\begin{definition}{Contradiction}
  A compound proposition that always evaluates to false; for example, $p \land \neg p$. (Comparable to a tautology) If a proposition is not a contradiction, it is then \textit{satisfyable}.
\end{definition}

\section*{2019-01-24}

\subsection*{Tautologies}

Which of these is a tautology? (three are)

\begin{itemize}
  \item $[(p \rightarrow q) \land (r \rightarrow q)] \rightarrow [(p \land r) \rightarrow q]$
  \item $[(p \rightarrow q) \lor  (r \rightarrow q)] \rightarrow [(p \land r) \rightarrow q]$
  \item $[(p \rightarrow q) \lor  (r \rightarrow q)] \rightarrow [(p \lor r) \rightarrow q]$
  \item $[(p \rightarrow q) \land  (r \rightarrow q)] \rightarrow [(p \lor r) \rightarrow q]$
\end{itemize}

Connectives are binary operators. This means that $a \land b \land c$ can be
read as either $(a \land b) \land c$ or $a \land (b \land c)$. It doesn't
matter as long as they're all the same operation. 

However, if they are \textit{not} all the same operation, this doesn't work: $a \land b \lor c$ has two non equivalent possibilities, $(a \land b) \lor c$ and $a \land (b \lor c)$.

\subsection*{Logical Equivalences}

\begin{definition}{Logically Equivalent}
  $S(p, q, r, \ldots)$ is said to be logically equivalent to $T(p, q, r, \ldots)$ whenever $S \leftrightarrow T$ is a tautology. This is written $S \equiv T$.
\end{definition}

Note: In all of the following cases, \textit{not} ($\neg$ is the first operation to be applied.

\begin{itemize}
  \item Associative Laws:
    $(a \land b) \land c \equiv a \land (b \land c)$ and
    $(a \lor b) \lor c \equiv a \lor (b \lor c)$, which means that parentheses
    can be dropped when order doesn't matter.
  \item Commutative Laws:
    $a \lor b \equiv b \lor a$ and $a \land b \equiv b \land a$
  \item Double Negation Law:
    $\neg(\neg a) \equiv a$
  \item De Morgan's Laws:
    $\neg(a \lor b) \equiv \neg a \land \neg b$ and
    $\neg(a \land b) \equiv \neg a \lor \neg b$
  \item Identity Laws:
    $a \land T \equiv a$ and
    $a \lor F \equiv a$
  \item Domination Laws:
    $a \land F \equiv F$ and
    $a \lor T \equiv T$
  \item Idempotent Laws
    $a \lor a \equiv a$ and
    $a \land a \equiv a$
  \item Distributive Laws:
    $a \land (b \lor c) \equiv (a \lor b) \land (a \lor c)$ and
    $a \lor (b \land c) \equiv (a \land b) \lor (a \land c)$
  \item Apparently-no-name Laws:
    $p \leftrightarrow q \equiv \neg (p \oplus q)$ and
    $p \rightarrow q \equiv \neg p \lor q$
\end{itemize}

Aside: $(p \land q) \lor (p \rightarrow q) \equiv p \rightarrow q$

$$(p \land q) \lor (p \rightarrow q)$$
$$\equiv (p \land q) \lor \neg p \lor q$$
$$\equiv \neg p \lor (p \land q) \lor q$$
$$\equiv [(\neg p \lor p) \land (\neg p \lor q)] \lor q$$
$$\equiv (\neg p \lor q) \lor q)$$
$$\equiv \neg p \lor q$$
$$\equiv p \rightarrow q$$

\end{document}
