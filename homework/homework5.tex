\documentclass{article} 

\title{Homework 5}
\author{Chandler Swift}
\date{February 19, 2019}
\usepackage{amsthm}
\usepackage{seqsplit}

\begin{document}
\maketitle
\begin{itemize}
    % To prove there is a unique, prove that there is at least one, and that
    % there is at most one.
    % 1.8 2, 4, 12, 16, 18, 24, 26, 32, 36
  \item[2] Use a proof by cases to show that 10 is not the square of a positive
    integer.
    \begin{proof}
      Case 1: Suppose $1 \leq x \leq 3$. None of the squares 1, 2, and 3 (1, 4, and 9) are equal to 10.
      Case 2: Suppose $x \geq 4$. All values of $x$ square to more than 10, with
      the smallest, 4, squaring to 16.
    \end{proof}
  \item[4] Prove that there are no positive perfect cubes less than 1000 that
    are the sum of the cubes of two positive integers.
    \begin{proof}
      Proof by exhaustion: \\
      For two positive numbers to sum to a thousand, they must both be smaller
      than 1000. There are nine positive perfect cubes less than a thousand:
      $1^3=1, 2^3=8, 3^3=27, 4^3=64, 5^3=125, 6^3=218, 7^3=343, 8^3=512, 9^3=729$.
      If one cube is $1^3=1$, then the other cube must be $1000-1=999$, which is
      not a perfect cube. If one cube is 8, then then the other must be 992, which
      is not a perfect cube. Similarly, the other perfect cubes would require
      complements of 963, 936, 875, 782, 657, 488, and 271; none of which are
      perfect cubes. All other cubes would produce numbers greater than a thousand.
    \end{proof}
  \item[12] Prove that either $2\cdot10^{500}$ or $2\cdot10^{500}$ is not a perfect square; that is, for big numbers, n and n+1 aren't perfect squares.
    \begin{proof}
      Let $x=$
      \seqsplit{14142135623730950488016887242096980785696718753769480731766797379907324784621070388503875343276415727350138462309122970249248360558507372126441214970999358314132226659275055927557999505011527820605714701095599716059702745345968620147285174186408891986}.

      Then $x^2=$
      \seqsplit{199999999999999999999999999999999999999999999999999999999999999999999999999999999999999999999999999999999999999999999999999999999999999999999999999999999999999999999999999999999999999999999999999999999999999999999999999999999999999999999999999999999997298193289999228009518895947571376931932785474994324356082085924607018660267610379324142405528110343824662274772915135086945882872259248936883583381612245882659082060547470527297162118365520965549759159552553981101339907371766857358706139448215024196}, 
      which is less than $2\cdot10^{500}+15$.

      Similarly, $(x+1)^2=$
      \seqsplit{200000000000000000000000000000000000000000000000000000000000000000000000000000000000000000000000000000000000000000000000000000000000000000000000000000000000000000000000000000000000000000000000000000000000000000000000000000000000000000000000000000000025582464537461128985552670431765338503326222982533285819615680684421668229509751156331893092080941798524939199391161075585442603989273993189766013323610962510923535379097582382413161128388576606761188561743753413220745398063704097653276487821032808169},
      which is more than $2\cdot10^{500}+16$. Thus neither is a perfect square.
    \end{proof} (This is a nonconstructive proof.)
  \item[16] Disprove: 2 and $\frac{1}{2}$ are both rational numbers,
    but $\sqrt{2}$ is not rational.
  \item[18] Show that if $a$, $b$, and $c$ are real numbers and $a \neq 0$,
    then there is a unique solution of the equation $ax+b=c$.
    \begin{proof}
      Suppose $a$, $b$, and $c$ are real numbers, with $a \neq 0$.
      Rearranging, we have $x=\frac{c-b}{a}$, a solution.

      Suppose there were two unique solutions, $x$ and $y$. Then
      $ax+b=c$ and $ay+b=c$, so $ax+b=ay+b$. Subtracting $b$ from both
      quantities leaves $ax=ay$, and dividing by $a$ leaves $x=y$, which
      means the solutions are not unique. Therefore there is eqactly one
      solution.
    \end{proof}
  \item[24] Show that if $x$ is a nonzero real number, then
    $x^2 + 1/x^2 \geq 2$.
    \begin{proof}
      Since $(x-1/x)^2 \geq 0$ (given), we can write this as
      $(x-1/x)^2 = (x-1/x)(x-1/x) \geq 0$. This, in turn, can be written as
      $x^2 - 2\not \frac{x}{x} + 1/x^2 \geq 0$, which is equivalent to
      $x^2 + 1/x^2 \geq 2$.
    \end{proof}
  \item[26]
    Conjecture: The quadratic mean is greater than or equal to the arithmetic mean
    when both numbers are positive.
    \begin{proof}
      Suppose there is a pair of numbers where the quadratic mean is less than the
      arithmetic mean. Then for $x,y$, $\sqrt{(x^2+y^2)/2}<(x+y)/2$.
      $$\sqrt{x^2+y^2} < (x+y)$$
      $$\sqrt{x^2+y^2} < \sqrt{x^2+2xy+y^2}$$
      Since x and y are positive, then 2xy is positive (proven previously),
      and the second root is greater than the first---a contradiction! 
      Therefore, the quadric mean is greater than or equal to the arithmetic
      mean for a pair of positive numbers.
    \end{proof}
  \item[32]
    \begin{proof}
      Because both addends are the product of a positive number and a squared
      integer, they must both be zero or positive. For two positive integers
      to sum to 14 they must be less than or equal to 14. If $|x| > 2$ or
      $|y| > 1$, then that addend will be greater than 14 and not satisfy the
      equation. If $y=0$, then no $|x|$ in ${0,1,2}$ satisfies the equation.
      Otherwise, if $|y|=1$, then no $|x|$ in ${0,1,2}$ satisfies the equation.
      If $|y| \geq 2$, then $5y^2 > 14$, so no solution can exist. Therefore,
      there is no solution.
    \end{proof}
  \item[36] Prove that $\sqrt[3]{2}$ is irrational.
    \begin{proof}
      A proof by contradiction: Suppose that the cube root of two. Then there
      exist some natural numbers $x,y$ for which $\frac{x}{y}=\sqrt[3]{2}$,
      and where the fraction is in its simplest form.
      Then $2=\frac{x^3}{y^3} \to 2y^3=x^3$. So $x^3$ (and therefore $x$) are
      divisible by 2, so $x=2k$ for some natural number $k$. By the same logic,
      $y$ must also be divisible by 2, and so they numbers have a common factor.
      However, we supposed they were in reduced form. Contradiction! Therefore,
      $\sqrt[3]{2}$ must be irrational.
    \end{proof}
\end{itemize}

\end{document}
