\documentclass{article} 

\title{Homework 10}
\author{Chandler Swift}
\date{April 2, 2019}
\usepackage{amsthm}
\usepackage{amsmath}
\usepackage{amsfonts}

\begin{document}
\maketitle
\begin{enumerate}
    % 5.2: 4, 6, 12, 30

%    \begin{proof} by way of mathematical induction:\\
%      When $n=1$, $1 \cdot 1! = 1 = (1+1)!-1$, which is true.
%
%      Now suppose that $1 \cdot 1! + 2 \cdot 2! + \dots + n \cdot n! = (n+1)! - 1$
%      for some $n$.
%      \begin{align*}
%        1 \cdot 1! + \dots + n \cdot n! + (n+1)(n+1)!
%        &= (n+1)! - 1 + (n+1)(n+1)!\\
%        &= 1(n+1)! + (n+1)(n+1)! - 1\\
%        &= (n+2)(n+1)! -1\\
%        &= (n+2)! - 1\\
%        &= ((n+1) + 1)! -1
%      \end{align*}
%
%      Since the $n+1$th case followed from the $n$th case, by the principle
%      of mathematical induction,
%      $1 \cdot 1! + 2 \cdot 2! + \dots + n \cdot n! = (n+1)! - 1$.
%    \end{proof}

  \item[4] Let $P(n)$ be the statement that a postage of $n$ cents can be
    formed using just 4-cent stamps and 7-cent stamps. The aprts of this
    exercise outline a strong induction proof that $P(n)$ is true for all
    integers $n \geq 18$. 
    \begin{enumerate}
      \item Show that the statements $P(18),P(19),P(20),P(21)$ are true,
        completing the basis step of a proof by strong induction.
        \begin{proof}
          For 18 cents of stamps, take two seven-cent stamps and one four-cent
          stamp.

          For 19 cents of stamps, take one seven-cent stamp and three four-cent
          stamps.

          For 20 cents of stamps, take five four-cent stamps.

          For 21 cents of stamps, take three seven-cent stamps.
        \end{proof}
      \item The inductive hypothesis of a proof by strong induction that $P(n)$
        is true for all integers $n \geq 18$ is that $P(1), P(2), \dots P(n-1)$
        are all true.
      \item In the inductive step, we need to prove that for some $P(n)$, if
        the inductive hypothesis holds, then $P(n)$ is true.
      \item
        \begin{proof}
          Suppose any amount 18 or more which is less than $k$ can be made up
          of stamps of denominations of four and seven cents. Since $k\geq21$,
          either $k=21$ or $k \geq 22$. If $k=21$, then $k$ can be made up of
          3 7-cent stamps. If $k \geq 22$, then $k-4 \geq 18$, and so by the
          inductive hypothesis can be made into a combination of four and seven
          cent stamps. Then $k$ an be made of the combination of $k-4$ and an
          additional four-cent stamp.
        \end{proof}
      \item Since the base cases were true, and all following cases were true
        if the base cases were true, then all cases greater than or equal to
        the base case must be true.
    \end{enumerate}
  \item[6]
    \begin{enumerate}
      \item The possible amounts, in cents, are 0, 3, 6, 9, 10, 12, 13, 15, 16,
        and $n \geq 18$.
      \item
        \begin{proof} by mathematical induction:
          For small amounts, the following amounts can be formed, represented
          as $a: (b,c)$, where $a$ is the amount of postage, and $b$ and $c$
          are the number of 3- and 10-cent stamps respectively: $0: (0,0),
          3: (1,0), 6: (2,0), 9: (3,0), 10: (0,1), 12: (4,0): 13: (1,1),
          15: (5,0), 16: (2,1)$.

          For amounts $n \geq 18$: In the base case, $18: (6,0)$.

          Now suppose that for some $k \geq 19$, $k$ can be formed from stamps
          of 3 and 10 cents. This must consist of at least 3 3-cent stamps
          or at least 2 10-cent stamps, since the maximum amount that can
          be made from less than 3 3-cent and 2 10-cent stamps is 16 cents.
          If there are 3 3-cent stamps, remove them and replace with a 10-cent
          stamp, incrementing the value by 1. Otherwise, there must be 2
          10-cent stamps. Remove these and replace with 7 3-cent stamps,
          also increasing the value by 1. Therefore $k+1$ cents can be formed
          if $k$ can.

          Since the $k+1$ case followed from the $k$, by the principle of
          mathematical induction, any number $n \geq 18$ can be formed from
          3- and 10-cent stamps.
        \end{proof}
      \item
        \begin{proof} by strong induction:

          For small amounts, the following amounts can be formed, represented
          as $a: (b,c)$, where $a$ is the amount of postage, and $b$ and $c$
          are the number of 3- and 10-cent stamps respectively: $0: (0,0),
          3: (1,0), 6: (2,0), 9: (3,0), 10: (0,1), 12: (4,0): 13: (1,1),
          15: (5,0), 16: (2,1)$.

          For amounts $n \geq 18$: In the base cases, $18: (6,0), 19: (3,1),
          20: (0,2)$.

          Now suppose that for some $k \geq 21$, all numbers $n | 18 \leq n
          < 21$ can be formed from three- and ten-cent stamps. \textit{
          (Inductive Hypothesis)} Since $k \geq 21$, then $k-3 \geq 18$, and
          so can be formed from three- and ten-cent stamps. Then $k$ can be
          formed from the combination making up $k-3$ plus one three-cent
          stamp.

          Since the $k$th case followed from the previous cases, by the
          principle of strong mathematical induction, any integer number
          of cents greater than or equal to 18 can be formed from three-
          and ten-cent stamps.
        \end{proof}
    \end{enumerate}
  \item[12]
    \begin{proof} by strong mathematical induction:
      
        In the base case, $1 = 2^0$.

        Now suppose that for some $k$, where for any $n | 1 \leq n \leq k$, $n$
        can be written as a sum of distinct powers of two. Then if $k+1$ is
        even, $\frac{k+1}{2}$ is an integer, and can be written as a sum of
        distinct powers of two $2^a + 2^b + 2^c\dots$. Then $k+1$ can be written
        as $2^{a+1} + 2^{b+1} + \cdots$, multiplying each power of two by two, 
        which means that they are still distinct.

        Now if $k+1 is odd$, then $k$ is even, and $k+1$ can be written as 
        $k + 2^0$, and since $k$ is even, it cannot contain $2^0$, so the sum
        is of distinct powers of two.

        Since the $k+1$ case followed from the previous cases, by the principle
        of strong mathematical induction, any positive integer can be written
        as a sum of distinct powers of two.
    \end{proof}
  \item[30]
    The induction step assumes that both $a^j = 1$ and $k > 0$. No base case
    can be shown where both of these are true.
\end{enumerate}

\end{document}
